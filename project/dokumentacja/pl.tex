\documentclass[12pt,a4paper,twoside]{article}

% ---------- POLISH LANGUAGE SUPPORT ----------
\usepackage[polish]{babel}     % correct hyphenation and translations
\usepackage[utf8]{inputenc}    % UTF-8 encoding for Polish characters
\usepackage[T1]{fontenc}       % proper Polish diacritics in PDF
\usepackage{lmodern}           % Latin Modern font (better PDF output)
\usepackage{csquotes}

% ---------- PAGE & TEXT SETTINGS ----------
\usepackage[margin=2.5cm]{geometry}
\usepackage{setspace}
\onehalfspacing                % 1.5 line spacing
\usepackage{indentfirst}       % indent first paragraph of each section
\usepackage{fancyhdr}
\usepackage{{titling}}

% ---------- MATH & GRAPHICS $ CIRCUITS ----------
\let\lll\relax
\usepackage{amsmath, amssymb}
\usepackage{graphicx}
\usepackage{caption}
\captionsetup{font=small}
\usepackage{subcaption}
\usepackage{siunitx}           % for units (e.g., \SI{10}{\volt})
\usepackage{standalone}        % for inluding pictures
\usepackage[american]{circuitikz}
\usepackage{pgfplots}
\pgfplotsset{compat=1.18}
\usepackage{booktabs}  % for nicer tables
\usepackage{siunitx}   % for aligning numbers by decimal point
\usepackage{diagbox}   % for diagonal cells
\usepackage{hhline}
\usepackage{multirow}
\usepackage{array}

% ---------- CUSTOM FLOATS ----------
\usepackage{float}
\newfloat{rysunek}{htbp}{lop}[section]
\floatname{rysunek}{Rysunek}

\newfloat{wykres}{htbp}{lop}[section]
\floatname{wykres}{Wykres}


% ---------- HYPERLINKS ----------
\usepackage[hidelinks]{hyperref}

% ---------- DOCUMENT INFO ----------

\title{Dokumentacja projektu: \\
    \textbf{Biblioteka do obsługi macierzy}}
\author{Autor projektu:\\
\textbf{Sylwester Dąbrowski}\\
Prowadzący:\\
\textbf{mgr inż. Jakub Zimnol}\\
Przedmiot:\\
Języki Programowania Objektowego
}
\date{\today}

% --------- Page style setup -----------------
\pagestyle{fancy}       % enable fancy headers/footers
\fancyhf{}              % clear all header and footer fields

% Footer
\fancyfoot[C]{\tiny{Biblioteka do obsługi macierzy - Sylwester Dąbrowski}}   % center of footer: 
\fancyfoot[LE,RO]{\thepage}

%Title page
\pretitle{\begin{center}\vspace*{-1cm}\Huge}   % title formatting
\posttitle{\end{center}
\begin{center}
\vfill
\includegraphics[width=0.35\textwidth]{rysunki/agh_logo.jpg}
\end{center}
\vfill}                           % fill space below title
\preauthor{\large\begin{center}}                                             % nothing before author
\postauthor{\end{center}}                               % vertical space
\predate{\begin{center}}                                                % nothing before date
\postdate{\end{center}}  



% ---------- BEGIN DOCUMENT ----------
\begin{document}



\maketitle 
\thispagestyle{empty}  %usuwanie numeru strony tytułowej
%\setcounter{page}{0}  % To sprawia liczenie od 1 następnej strony

\newpage

\section{Opis projektu}

Projekt przedstawia implementację prostej biblioteki do obsługi macierzy matematycznych w języku \texttt{C++},
z~wykorzystaniem mechanizmów programowania generycznego (szablonów). Biblioteka umożliwia tworzenie i~wykonywanie operacji
na macierzach prostokątnych, kwadratowych oraz trójkątnych, z~zachowaniem poprawnych własności matematycznych i kontroli błędów.

Celem projektu było zaprojektowanie czytelnego i~bezpiecznego interfejsu umożliwiającego wykonywanie podstawowych operacji algebraicznych na macierzach, 
takich jak dodawanie, odejmowanie, mnożenie, transpozycja, obliczanie wyznacznika oraz macierzy odwrotnej.

Biblioteka może znaleźć zastosowanie w~prostych obliczeniach numerycznych, zadaniach dydaktycznych z~algebry liniowej 
oraz jako baza do dalszej rozbudowy (np. o~bardziej zaawansowane algorytmy numeryczne).

\section{Struktura i opis klas}

Projekt składa się z~trzech głównych klas, zorganizowanych hierarchicznie:

\begin{itemize}
    \item \texttt{Matrix<T>} -- klasa bazowa reprezentująca ogólną macierz prostokątną o~dowolnych wymiarach. 
    Udostępnia podstawowe operacje na macierzach oraz mechanizmy dostępu do elementów.
    
    \item \texttt{SquareMatrix<T>} -- klasa dziedzicząca po \texttt{Matrix<T>}, przeznaczona do obsługi macierzy kwadratowych.
     Rozszerza funkcjonalność o~operacje charakterystyczne dla macierzy kwadratowych, takie jak obliczanie wyznacznika, 
     macierzy dopełnień algebraicznych oraz macierzy odwrotnej.
    
    \item \texttt{TriangularMatrix<T>} -- klasa dziedzicząca po \texttt{SquareMatrix<T>}, 
    reprezentująca macierze trójkątne (dolne lub górne). Klasa wymusza zachowanie struktury
     trójkątnej poprzez kontrolę zapisu elementów oraz oferuje zoptymalizowane obliczanie wyznacznika.
\end{itemize}

Relacje dziedziczenia odzwierciedlają zależności matematyczne pomiędzy typami macierzy, co pozwala na ponowne wykorzystanie kodu i~zachowanie 
spójności interfejsu.

\section{Uruchomienie i kompilacja}

Projekt wykorzystuje system budowania \texttt{CMake}. Do skompilowania aplikacji wymagany jest kompilator obsługujący standard \texttt{C++20}.

Sekwencja poleceń umożliwiająca poprawne zbudowanie projektu:

\begin{verbatim}
mkdir build
cd build
cmake ..
make
\end{verbatim}

W wyniku kompilacji powstaje plik wykonywalny \texttt{matrixSD}, który zawiera program demonstracyjny prezentujący możliwości zaimplementowanej biblioteki.

\section{Dodatkowe informacje}

Kod projektu został podzielony na pliki nagłówkowe i źródłowe, a jego dokumentacja techniczna została przygotowana w formacie Doxygen bezpośrednio w plikach nagłówkowych.

Plik \texttt{main.cpp} pełni rolę programu testowego, demonstrującego poprawność działania zaimplementowanych klas, obsługę wyjątków oraz przykładowe obliczenia wykonane 
za pomocą funkcji z~biblioteki.

\end{document}